
%%%%%%%%%%%%%%%%%%%%%% Feynman diagram for T6qqWG

\documentclass{article}

\input{shared/header.tex}


%%%%%%%%%%%%%%%%%%%%%%%%%%% Document %%%%%%%%%%%%%%%%%%%%%%%%%%%
\begin{document}
\thispagestyle{empty}


%%%%%%%% THE NAME OF THE fmffile HAS TO BE ``Feynman<filename>'' TO USE compile.py %%%%%%%%%%%%%%%
\begin{fmffile}{FeynmanT6qqWG}
\parbox{300mm}{

\begin{fmfgraph*}(180,90) %\fmfpen{thick}

  \fmfset{arrow_len}{cm}\fmfset{arrow_ang}{0}
  %%%% change width of wiggle to make EWKino lines visible
  \fmfset{wiggly_slope}{75}

  %%%%%%%%%%%% Specifying number of inputs/outputs
  \fmfleftn{i}{2}
  \fmfrightn{o}{6}
  \fmflabel{}{i1}
  \fmflabel{}{i2}

  %%%%%%%%%%%% Incoming protons (one line)
  \fmf{fermion, tension=2, lab=p, label.side=right}{v1,i1}
  \fmf{fermion, tension=2, lab=p, label.side=left}{v1,i2}

  %%%%%%%%%%%% Produced SUSY particles
  \fmf{dashes, label=\sQ, tension=1.5, label.side=left, label.dist=8.}{v1,v3}
  \fmf{dashes, label=\sQb, tension=1.5, label.side=right, label.dist=10.}{v1,v2}
  %%%%%%%%%%%%% Decays and vertex circles

  \fmf{fermion}{v2,o1}
  \fmflabel{\anti{q}}{o1}

  \fmf{plain, foreground=black, label=\chipm, label.side=right, label.dist=-13}{v2,v4}
  \fmf{boson, foreground=black}{v2,v4}
  \fmf{photon}{v4,o2}

  \fmflabel{\sGra}{o3}
  \fmf{dots, foreground=black}{v4,o3}


  \fmf{fermion}{v3,o6}
  \fmflabel{q}{o6}

  \fmf{plain, foreground=black, label=\chiz, label.side=right, label.dist=2}{v3,v5}
  \fmf{boson, foreground=black}{v3,v5}
  \fmf{photon}{v5,o5}
  \fmflabel{$\gamma$}{o5}
  \fmflabel{\PW}{o2}

  \fmflabel{\sGra}{o4}
  \fmf{dots, foreground=black}{v5,o4}

  %% Vertex circles
  \fmfdot{v2,v3,v4,v5}

  %%%%%%%%%%%% Additional lines on incoming protons and blob
  \input{shared/protons.tex}


\end{fmfgraph*}

}
\end{fmffile}

\end{document}
