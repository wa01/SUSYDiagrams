
%%%%%%%%%%%%%%%%%%%%%% Feynman diagram for T1bbbb

\documentclass{article}

\input{shared/header.tex}

\def\MainQuark{b}


%%%%%%%%%%%%%%%%%%%%%%%%%%% Document %%%%%%%%%%%%%%%%%%%%%%%%%%%
\begin{document}
\thispagestyle{empty}


%%%%%%%% THE NAME OF THE fmffile HAS TO BE ``Feynman<filename>'' TO USE compile.py %%%%%%%%%%%%%%%
\begin{fmffile}{FeynmanT1bbbb}
\parbox{300mm}{

\begin{fmfgraph*}(180,90)
  \fmfset{arrow_len}{cm}\fmfset{arrow_ang}{0}
  %%%% change width of wiggle to make EWKino lines visible
  \fmfset{wiggly_slope}{75}
  
  %%%%%%%%%%%% Specifying number of inputs/outputs
  \fmfleftn{i}{2}
  \fmfrightn{o}{6}
  \fmflabel{}{i1}
  \fmflabel{}{i2}
    
  %%%%%%%%%%%% Incoming protons (one line)
  \fmf{fermion,  tension=2.8, lab=p, label.side=right}{v1,i1}
  \fmf{fermion,  tension=2.8, lab=p, label.side=left}{v1,i2}
           
  %%%%%%%%%%%% Produced SUSY particles
  \fmf{gluon, label=\sGlu,label.dist=+10, label.side=left}{v1,v2} %upper vertex
  \fmf{gluon, label=\sGlu,label.dist=+12}{v1,v3} %lower vertex
  \fmf{fermion}{v1,v2}
  \fmf{fermion}{v1,v3}


  %%%%%%%%%%%%% Decays and vertex circles
  \fmflabel{$\mathrm{\overline{\MainQuark}}$}{o1}
  \fmflabel{\MainQuark}{o2}
  \fmflabel{\chiz}{o3}
  \fmf{fermion}{o1,v3,o2}
  \fmf{plain}{v3,o3}
  \fmf{boson}{v3,o3}

  %% 2nd decay
  \fmflabel{\chiz}{o4}
  \fmflabel{$\mathrm{\overline{\MainQuark}}$}{o5}
  \fmflabel{\MainQuark}{o6}
  \fmf{fermion}{o5,v2,o6}
  \fmf{plain}{v2,o4}
  \fmf{boson}{v2,o4}

  %% Vertex circles
  \fmfdot{v2,v3}
           
  %%%%%%%%%%%% Additional lines on incoming protons
  \input{shared/protons.tex}

\end{fmfgraph*}
       
}           
\end{fmffile} 

\end{document}
