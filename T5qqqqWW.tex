
%%%%%%%%%%%%%%%%%%%%%% Feynman diagram for T5qqqqWW

\documentclass{article}

\input{shared/header.tex}

\def\MainQuark{q}


%%%%%%%%%%%%%%%%%%%%%%%%%%% Document %%%%%%%%%%%%%%%%%%%%%%%%%%%
\begin{document}
\thispagestyle{empty}


%%%%%%%% THE NAME OF THE fmffile HAS TO BE ``Feynman<filename>'' TO USE compile.py %%%%%%%%%%%%%%%
\begin{fmffile}{FeynmanT5qqqqWW}
\parbox{300mm}{

\begin{fmfgraph*}(180,90)
  \fmfset{arrow_len}{cm}\fmfset{arrow_ang}{0}
  %%%% change width of wiggle to make EWKino lines visible
  \fmfset{wiggly_slope}{75}
  
  %%%%%%%%%%%% Specifying number of inputs/outputs
  \fmfleftn{i}{2}
  \fmfrightn{o}{8}
  \fmflabel{}{i1}
  \fmflabel{}{i2}
    
  %%%%%%%%%%%% Incoming protons (one line)
  \fmf{fermion,  tension=2.8, lab=p, label.side=right}{v1,i1}
  \fmf{fermion,  tension=2.8, lab=p, label.side=left}{v1,i2}
           
  %%%%%%%%%%%% Produced SUSY particles
  \fmf{gluon, label=\sGlu,label.dist=+10, label.side=left}{v1,v2} %upper vertex
  \fmf{gluon, label=\sGlu,label.dist=+12, label.side=right}{v1,v3} %lower vertex
  \fmf{fermion}{v1,v2}
  \fmf{fermion}{v1,v3}


  %%%%%%%%%%%%% Decays and vertex circles
  \fmflabel{$\mathrm{\overline{\MainQuark}^\prime}$}{o1}
  \fmflabel{\MainQuark}{o2}
  \fmflabel{\PW}{o3}
  \fmflabel{\chiz}{o4}
  \fmf{fermion}{o1,v3,o2}
  \fmf{plain, label=\chipm, label.dist=+3, label.side=left}{v3,v4}
  \fmf{boson}{v3,v4}
  \fmf{plain}{v4,o4}
  \fmf{boson}{v4,o4}
  \fmf{photon}{v4,o3}

  %% 2nd decay
  \fmflabel{\chiz}{o5}
  \fmflabel{\PW}{o6}
  \fmflabel{$\mathrm{\overline{\MainQuark}^\prime}$}{o7}
  \fmflabel{\MainQuark}{o8}
  \fmf{fermion}{o7,v2,o8}
  \fmf{plain, label=\chipm, label.dist=+2, label.side=right}{v2,v5}
  \fmf{boson}{v2,v5}
  \fmf{plain}{v5,o5}
  \fmf{boson}{v5,o5}
  \fmf{photon}{v5,o6}


  %% Vertex circles
  \fmfdot{v2,v3,v4,v5}
           
  %%%%%%%%%%%% Additional lines on incoming protons
  \input{shared/protons.tex}

\end{fmfgraph*}
       
}           
\end{fmffile} 

\end{document}
